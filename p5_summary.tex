    \section{总结}
    \subsection{总结}

    \begin{frame}
      \frametitle{总结}
      \begin{itemize}
        \item 基于图像增强的方法不考虑有雾图像的形成过程,而是直接通过突出图像的细节,提高对比度等方式,从而使有雾图像看上去更加清晰。
        \item 基于物理模型的方法则是追寻图像降质的物理过程,通过物理模型还原出清晰的图像。
        \item 基于深度学习的方法则是利用神经网络强大的学习能力,寻找有雾图像与图像复原物理模型中某些系数的映射关系或者使用 GAN,根据有雾图像还原出无雾的清晰图像。
      \end{itemize}
    \end{frame}

    \begin{frame}
      \frametitle{总结}
      上述三类去雾算法对于雾天图像都有着明显的去雾效果,尽管其在实际生活中已经得到了广泛的应用,但下述几点仍有可能是今后图像去雾领域的研究重点和难点:
      \begin{itemize}
        \item 更加真实的雾天图像数据集
        \item 更加简便的去雾算法
        \item 鲁棒性更强的去雾算法
      \end{itemize}
    \end{frame}

    \begin{frame}
      
    \end{frame}

    \frame{
      \frametitle{ }
      
       ~\\ ~\\
       \center{\Large{Thank you!}}
       \\ ~\\ ~\\ ~\\ ~\\ 

    }
